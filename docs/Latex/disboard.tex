\documentclass{article}
\textheight=8.5in
\textwidth=6.5in


\pagestyle{empty}                            % no headers and page numbers
\oddsidemargin -10 true pt      % Left margin on odd-numbered pages.
\evensidemargin 10 true pt      % Left margin on even-numbered pages.
\marginparwidth 0.75 true in    % Width of marginal notes.
\oddsidemargin  0 true in       % Note that \oddsidemargin=\evensidemargin
\evensidemargin 0 true in
\topmargin -0.75 true in        % Nominal distance from top of page to top of\textheight 9.5 true in         % Height of text (including footnotes and figures)
\textwidth 6.375 true in        % Width of text line.
\parindent=0pt                  % Do not indent paragraphs
\parskip=0.15 true in
\usepackage{color}              % Need the color package

  

\begin{document} 
\begin{center}
\section*{\underline{Discussion Board}}
                   Author Name : Suman Rajput \& Rekha Pal
\end{center}
\subsection*{\underline{Sending discussion}}
\subsubsection*{Parameters: User-id, group-id, subject attachment, expiry}
Send request for send a discussion.
Add the discussion to the specfic group with specfic message.
It is sucessfully send the message "Discussion send sucessfully" else "same error in DiscussionBoard".

\subsection*{\underline{Update Discussion}}
\begin{enumerate}
\item Send a request for update the discussion send the updated details along with the request.
\item Update the details if it is succesfuly send the message"Discussion Updated sucessfully" else "same error in Updation of DiscussionBoard".
\begin{center}
\input{Updatedis.latex}
\label{figure:Updatedis.latex}
\end{center}
\end{enumerate}
\section*{\underline{Deletion Discussion}}
\begin{enumerate}
\item  Send a request for deleting the Discussion.
\item Delete the Discussion to the specific group with specific message.
\item If step (2)is not successful than send the message"Deletion not successfully".
\begin{center}
\input{Deletiondis.latex}
\label{figure:Deletiondis.latex}
\end{center}
\end{enumerate}
\subsection*{\underline{Expiry enable/disable}}
\begin{enumerate}
\item User Login
\item Sending request to enable/disable the message.
\item Return the appropriate message.
Parameter=Msg-id
\begin{center}
\input{Expirydis.latex}
\label{figure:Expirydis.latex}
\end{center}
\end{enumerate}
\subsection*{\underline{Permission Given-Deny}}
Permission for discussion,We search for that msg-id of that is permitted from DB\_SEND table.
User Login Sending request to permission given/deny the message.
\input{Permissiondis.latex}
\label{figure:Permissiondis.latex}
\subsection*{\underline{Backup of Discussion}}

Send request for backup a specific discussion.
Parameters:Msg-id
Store the Whole description of discussion with specific message.
\input{Backupdis.latex}
\label{figure:Backupdis.latex}
\subsubsection*{\underline{Reply Discussion}}
\begin{enumerate}
\item Send request for reply of specific discussion.
      Parameter: Msg-id,User-id,Group-id,Subject,Text,attachment any expiry .
\item Add the reply of a specific discussion to the specific group with the specific message.
\item Return the appropriate message.
\begin{center}
\input{Replydis.latex}
\label{figure:Replydis.latex}
\end{center}
\end{enumerate}
\subsubsection*{\underline{View Discussion}}
\begin{enumerate}
\item Send request for viewing discussion of specific discussion parameter,msg-id.
\item Show all description of specific discussion.
\item Return the appropriate message.
\begin{center}
\input{viewdis.latex}
\label{figure:viewdis.latex}
\end{center}
\end{enumerate}
Written By :- Rekha Pal

\end{document}
