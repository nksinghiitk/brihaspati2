\documentclass{article}

\textheight=8.5in
\textwidth=6.5in
\oddsidemargin=0.0in
\topmargin=0.0in

\renewcommand\baselinestretch{1.0}           % single space
\pagestyle{empty}                            % no headers and page numbers
\oddsidemargin -10 true pt      % Left margin on odd-numbered pages.
\evensidemargin 10 true pt      % Left margin on even-numbered pages.
\marginparwidth 0.75 true in    % Width of marginal notes.
\oddsidemargin  0 true in       % Note that \oddsidemargin=\evensidemargin
\evensidemargin 0 true in
\topmargin -0.75 true in        % Nominal distance from top of page to top of\textheight 9.5 true in         % Height of text (including footnotes and figures)
\textwidth 6.375 true in        % Width of text line.
\parindent=0pt                  % Do not indent paragraphs
\parskip=0.15 true in
\usepackage{color}              % Need the color package
\usepackage{epsfig}

\begin{document}
\begin{center}
\section*{\underline{REPOSITORY}}
             Author Name : Manorama Pal \& Kishore Kumar Shukla

\input{Repository.latex}
\end{center}

\subsection*{\underline{CREATE REPOSITORY}}
\subsubsection*{\underline{USER}}
\begin{enumerate}
                 \item Interacte with web-application.
                       \item User send request to the admin to registerd as a author.
                          \item Successful message come to admin,user registered as a author \& create his own  Repository.
                             \item User management area where author registration done.
                            
\begin{center}
\input{Create_rep.latex}
\label{figure:Create_rep.latex}
\end{center}
\end{enumerate}



\subsubsection*{\underline{AUTHOR}}
          
  \begin{enumerate}
                     \item Author login as a content author \& goes to author home.
                          \item Author create his own  Repository.
                              \item After successful creation of repository author works his repository home.
                                  \item Return the apporopiate message.


\begin{center}
\input{createauthor_rep.latex}
\label{figure:createauthor_rep.latex}
\end{center}
\end{enumerate}

\subsection*{\underline{UPLOAD CONTENT}}

  \begin{enumerate}
                                \item Author login as a content author.
                                   \item Author send the request to upload content in existing topic or new topic.
                                     \item Return the appropriate message.
                                     \item Author works in his repository home.
\begin{center}
\input{Uploadcon_rep.latex}
\label{figure:Uploadcon_rep.latex}
\end{center}
\end{enumerate} 
\subsection*{\underline{VIEW CONTENT}}
\begin{enumerate}

                                  \item Author login as a content author.
                                       \item Sending request to view a file.
                                          \item Show the view file.
                                             \item Return the appropriate message.
                                                \item Repository home.
\begin{center}
\input{viewcon_rep.latex}
\label{figure:viewcon_rep.latex}
\end{center}
\end{enumerate}
\subsection*{\underline{DELETE DIR \& TOPIC}}
\begin{enumerate}
                          \item Author login as a content author.
                            \item Sending request to delete dir \& topic.
                              \item Delete dir \& topic.
                                \item Return the successful message.
                                   \item Repository home.  

\begin{center}
\input{deletedir_rep.latex}
{DELETE DIR \& TOPIC}
\label{figure:deletedir_rep.latex}
\end{center}
\end{enumerate}
\subsection*{\underline{MOVE FILE}}
\begin{enumerate}
                                \item Author login as a content author.
                                   \item Sending request to move file to another directory.
                                     \item Selected file to another directory.
                                       \item Return the successful message.
                                         \item Repository home.
\begin{center}
\input{move_rep.latex}
\label{figure:move_rep.latex}
\end{center}
\end{enumerate}
\subsection*{\underline{PERMISSION GIVEN}}
\begin{enumerate}
                       \item Author login as a content author.
                          \item Sending request to give permission.
                            \item Check Role(check the role i.e author,Instructor(Private area),Instructor(Course area))
      \item If role successful permission receive his area i.e authors's phase,Instructor(Private area)Instructor(Course area). 
                              \item If role check successful permission given topic view author's phase.
                                \item If failure role check given topic is not received.
                                  \item Return the appropriate message.
\begin{center}
\input{permission_rep.latex}
\label{figure:permission_rep.latex}
\end{center}
\end{enumerate}
\subsection*{\underline{REPOSITORY BROWSER}}
\begin{enumerate}
\item Author login as a content author.
\item Author view all the dir \& topic who registered as an author but not read the file.
\item Return the appropriate message.
\item Repository Browser.
\begin{center}
\input{repositoryBro_rep.latex}
\label{figure:repositoryBro_rep.latex}
\end{center}
\end{enumerate}
\subsection*{\underline{VIEW \& DELETE PERMISSION}}
\begin{enumerate}
\item Author login as a content author.
\item Author goes to his repository home.
\item Author send request to delete Permission Receive topic so Simultaneously delete Permission given topic in author phase.
\item Similarly author delete Permission given topic in author phase so Simultaneously delete Permission Receive topic in Author phase,Instructor(Private area) and Instructor(Course area).
\item Author,Instructor(Private area) and Instructor(Course area)request to view the Receive file.
\item View the Receive file.
\end{enumerate} 
\subsection*{\underline{MODIFY IN REPOSITORY}}
\begin{enumerate}
\item User login as an instructor and goes into the course management.
\item Instructor select the course content will select the publish link.
\item Permitted topic will display in the unpublished area.Instructor view those file.
\item Instructor will publish the selected files.
\item Instructor \& student both view the permitted file.
\begin{center}
\input{modify_rep.latex}
{MODIFY IN REPOSITORY}
\label{figure:modify_rep.latex}
\end{center}
\end{enumerate}
Written by :- Rekha Pal                                                   
\end{document}

