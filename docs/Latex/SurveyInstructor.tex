\documentclass{article}
\usepackage{graphicx}
\textheight=8.5in
\textwidth=6.5in
\oddsidemargin=0.0in
\topmargin=0.0in


\renewcommand\baselinestretch{1.0}           % single space
\pagestyle{empty}                            % no headers and page numbers
\oddsidemargin -10 true pt      % Left margin on odd-numbered pages.
\evensidemargin 10 true pt      % Left margin on even-numbered pages.
\marginparwidth 0.75 true in    % Width of marginal notes.
\oddsidemargin  0 true in       % Note that \oddsidemargin=\evensidemargin
\evensidemargin 0 true in
\topmargin -0.75 true in        % Nominal distance from top of page to top of\textheight 9.5 true in
       % Height of text (including footnotes and figures)
\textwidth 6.375 true in        % Width of text line.
\parindent=0pt                  % Do not indent paragraphs
\parskip=0.15 true in
\usepackage{color}              % Need the color package
\usepackage{epsfig}


\begin{document}
\begin{center}
\section*{\underline{SURVEY OF INSTRUCTOR}}
                             Author Name : Jaivir
\end{center}
\subsection*{\underline{ADMIN INTERFACE}}
\begin{enumerate}
\item Login as Admin.
\item Admin creates or chooses survey file and post the survey from to student for all courses.

\begin{center}
\input{Admin_ser.latex}
\label{figure:Admin_ser.latex}
\end{center}
\end{enumerate}
\subsection*{\underline{STUDENT INTERFACE}}
\begin{enumerate}
\item Login as Student.
\item student gose to courses and get the survey file.
\item student filled and submit the survey file and result stored in the survey result. Result stored in xml file containing the following attribute.
<instructor name, course name, year performance>
\begin{center}
\input{studentinter_ser.latex}
\label{figure:studentinter_ser.latex}
\end{center}
\end{enumerate}
\subsection*{\underline{DATA BASE}}
\subsubsection*{\underline{Table Name - Instructor Rating}}
\begin{tabular}{|r|r|r|r|r|r|}
 \hline
Instructor name & Course name  & Student name & No.of students & Rating 1,2,3,4,5 & Mean  \\  \hline
&&&&& \\ \hline
&&&&& \\ \hline
\end{tabular}
\begin{enumerate}
\item Student gives the marks as question user find rank of that Instructored is calculated by the about formula. by that student.
\item This means value is stored in the survey table is finally calculated the avg. value of particular instructor.
\end{enumerate}
\subsection*{\underline{TABLE NAME-SURVEY}}
 \begin{tabular}{|r|r|r|r|r|}
 \hline
Instructor name & Course name  & No. of students & Rating 1,2,3,4,5   \\  \hline
&&& \\ \hline
&&& \\ \hline
\end{tabular}
\begin{enumerate}
\item The over rating (all questions) given by every student is stored in the rating filed and calculates the avg. rating of that Instructor.
\item The avg. rating (performance) is stored in the xml file.
\end{enumerate}
\subsection*{\underline{RATING TYPE}}
\begin{enumerate}
 \item Avg 80 excellent.\\
 \item 80 Avg. 65 very good.\\
 \item Avg  65 good.\\
\end{enumerate}
\subsection*{\underline{FUNCTIONALITY OF ADMIN}}
\begin{enumerate}
\item Admin can modify the question and added extra question.
\item Admin send the survey result after 15 days of end of the semester to the Instructor.
\item Admin can delete the Survey result when he wants or survey result automatically delete to the end of the next Sem.
\end{enumerate}
Written by :- Rekha Pal
\end{document} 
